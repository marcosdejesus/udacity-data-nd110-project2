
% Default to the notebook output style

    


% Inherit from the specified cell style.




    
\documentclass[11pt]{article}

    
    
    \usepackage[T1]{fontenc}
    % Nicer default font (+ math font) than Computer Modern for most use cases
    \usepackage{mathpazo}

    % Basic figure setup, for now with no caption control since it's done
    % automatically by Pandoc (which extracts ![](path) syntax from Markdown).
    \usepackage{graphicx}
    % We will generate all images so they have a width \maxwidth. This means
    % that they will get their normal width if they fit onto the page, but
    % are scaled down if they would overflow the margins.
    \makeatletter
    \def\maxwidth{\ifdim\Gin@nat@width>\linewidth\linewidth
    \else\Gin@nat@width\fi}
    \makeatother
    \let\Oldincludegraphics\includegraphics
    % Set max figure width to be 80% of text width, for now hardcoded.
    \renewcommand{\includegraphics}[1]{\Oldincludegraphics[width=.8\maxwidth]{#1}}
    % Ensure that by default, figures have no caption (until we provide a
    % proper Figure object with a Caption API and a way to capture that
    % in the conversion process - todo).
    \usepackage{caption}
    \DeclareCaptionLabelFormat{nolabel}{}
    \captionsetup{labelformat=nolabel}

    \usepackage{adjustbox} % Used to constrain images to a maximum size 
    \usepackage{xcolor} % Allow colors to be defined
    \usepackage{enumerate} % Needed for markdown enumerations to work
    \usepackage{geometry} % Used to adjust the document margins
    \usepackage{amsmath} % Equations
    \usepackage{amssymb} % Equations
    \usepackage{textcomp} % defines textquotesingle
    % Hack from http://tex.stackexchange.com/a/47451/13684:
    \AtBeginDocument{%
        \def\PYZsq{\textquotesingle}% Upright quotes in Pygmentized code
    }
    \usepackage{upquote} % Upright quotes for verbatim code
    \usepackage{eurosym} % defines \euro
    \usepackage[mathletters]{ucs} % Extended unicode (utf-8) support
    \usepackage[utf8x]{inputenc} % Allow utf-8 characters in the tex document
    \usepackage{fancyvrb} % verbatim replacement that allows latex
    \usepackage{grffile} % extends the file name processing of package graphics 
                         % to support a larger range 
    % The hyperref package gives us a pdf with properly built
    % internal navigation ('pdf bookmarks' for the table of contents,
    % internal cross-reference links, web links for URLs, etc.)
    \usepackage{hyperref}
    \usepackage{longtable} % longtable support required by pandoc >1.10
    \usepackage{booktabs}  % table support for pandoc > 1.12.2
    \usepackage[inline]{enumitem} % IRkernel/repr support (it uses the enumerate* environment)
    \usepackage[normalem]{ulem} % ulem is needed to support strikethroughs (\sout)
                                % normalem makes italics be italics, not underlines
    

    
    
    % Colors for the hyperref package
    \definecolor{urlcolor}{rgb}{0,.145,.698}
    \definecolor{linkcolor}{rgb}{.71,0.21,0.01}
    \definecolor{citecolor}{rgb}{.12,.54,.11}

    % ANSI colors
    \definecolor{ansi-black}{HTML}{3E424D}
    \definecolor{ansi-black-intense}{HTML}{282C36}
    \definecolor{ansi-red}{HTML}{E75C58}
    \definecolor{ansi-red-intense}{HTML}{B22B31}
    \definecolor{ansi-green}{HTML}{00A250}
    \definecolor{ansi-green-intense}{HTML}{007427}
    \definecolor{ansi-yellow}{HTML}{DDB62B}
    \definecolor{ansi-yellow-intense}{HTML}{B27D12}
    \definecolor{ansi-blue}{HTML}{208FFB}
    \definecolor{ansi-blue-intense}{HTML}{0065CA}
    \definecolor{ansi-magenta}{HTML}{D160C4}
    \definecolor{ansi-magenta-intense}{HTML}{A03196}
    \definecolor{ansi-cyan}{HTML}{60C6C8}
    \definecolor{ansi-cyan-intense}{HTML}{258F8F}
    \definecolor{ansi-white}{HTML}{C5C1B4}
    \definecolor{ansi-white-intense}{HTML}{A1A6B2}

    % commands and environments needed by pandoc snippets
    % extracted from the output of `pandoc -s`
    \providecommand{\tightlist}{%
      \setlength{\itemsep}{0pt}\setlength{\parskip}{0pt}}
    \DefineVerbatimEnvironment{Highlighting}{Verbatim}{commandchars=\\\{\}}
    % Add ',fontsize=\small' for more characters per line
    \newenvironment{Shaded}{}{}
    \newcommand{\KeywordTok}[1]{\textcolor[rgb]{0.00,0.44,0.13}{\textbf{{#1}}}}
    \newcommand{\DataTypeTok}[1]{\textcolor[rgb]{0.56,0.13,0.00}{{#1}}}
    \newcommand{\DecValTok}[1]{\textcolor[rgb]{0.25,0.63,0.44}{{#1}}}
    \newcommand{\BaseNTok}[1]{\textcolor[rgb]{0.25,0.63,0.44}{{#1}}}
    \newcommand{\FloatTok}[1]{\textcolor[rgb]{0.25,0.63,0.44}{{#1}}}
    \newcommand{\CharTok}[1]{\textcolor[rgb]{0.25,0.44,0.63}{{#1}}}
    \newcommand{\StringTok}[1]{\textcolor[rgb]{0.25,0.44,0.63}{{#1}}}
    \newcommand{\CommentTok}[1]{\textcolor[rgb]{0.38,0.63,0.69}{\textit{{#1}}}}
    \newcommand{\OtherTok}[1]{\textcolor[rgb]{0.00,0.44,0.13}{{#1}}}
    \newcommand{\AlertTok}[1]{\textcolor[rgb]{1.00,0.00,0.00}{\textbf{{#1}}}}
    \newcommand{\FunctionTok}[1]{\textcolor[rgb]{0.02,0.16,0.49}{{#1}}}
    \newcommand{\RegionMarkerTok}[1]{{#1}}
    \newcommand{\ErrorTok}[1]{\textcolor[rgb]{1.00,0.00,0.00}{\textbf{{#1}}}}
    \newcommand{\NormalTok}[1]{{#1}}
    
    % Additional commands for more recent versions of Pandoc
    \newcommand{\ConstantTok}[1]{\textcolor[rgb]{0.53,0.00,0.00}{{#1}}}
    \newcommand{\SpecialCharTok}[1]{\textcolor[rgb]{0.25,0.44,0.63}{{#1}}}
    \newcommand{\VerbatimStringTok}[1]{\textcolor[rgb]{0.25,0.44,0.63}{{#1}}}
    \newcommand{\SpecialStringTok}[1]{\textcolor[rgb]{0.73,0.40,0.53}{{#1}}}
    \newcommand{\ImportTok}[1]{{#1}}
    \newcommand{\DocumentationTok}[1]{\textcolor[rgb]{0.73,0.13,0.13}{\textit{{#1}}}}
    \newcommand{\AnnotationTok}[1]{\textcolor[rgb]{0.38,0.63,0.69}{\textbf{\textit{{#1}}}}}
    \newcommand{\CommentVarTok}[1]{\textcolor[rgb]{0.38,0.63,0.69}{\textbf{\textit{{#1}}}}}
    \newcommand{\VariableTok}[1]{\textcolor[rgb]{0.10,0.09,0.49}{{#1}}}
    \newcommand{\ControlFlowTok}[1]{\textcolor[rgb]{0.00,0.44,0.13}{\textbf{{#1}}}}
    \newcommand{\OperatorTok}[1]{\textcolor[rgb]{0.40,0.40,0.40}{{#1}}}
    \newcommand{\BuiltInTok}[1]{{#1}}
    \newcommand{\ExtensionTok}[1]{{#1}}
    \newcommand{\PreprocessorTok}[1]{\textcolor[rgb]{0.74,0.48,0.00}{{#1}}}
    \newcommand{\AttributeTok}[1]{\textcolor[rgb]{0.49,0.56,0.16}{{#1}}}
    \newcommand{\InformationTok}[1]{\textcolor[rgb]{0.38,0.63,0.69}{\textbf{\textit{{#1}}}}}
    \newcommand{\WarningTok}[1]{\textcolor[rgb]{0.38,0.63,0.69}{\textbf{\textit{{#1}}}}}
    
    
    % Define a nice break command that doesn't care if a line doesn't already
    % exist.
    \def\br{\hspace*{\fill} \\* }
    % Math Jax compatability definitions
    \def\gt{>}
    \def\lt{<}
    % Document parameters
    \title{Projeto II - Fundamentos de Data Science I}
    
    
    

    % Pygments definitions
    
\makeatletter
\def\PY@reset{\let\PY@it=\relax \let\PY@bf=\relax%
    \let\PY@ul=\relax \let\PY@tc=\relax%
    \let\PY@bc=\relax \let\PY@ff=\relax}
\def\PY@tok#1{\csname PY@tok@#1\endcsname}
\def\PY@toks#1+{\ifx\relax#1\empty\else%
    \PY@tok{#1}\expandafter\PY@toks\fi}
\def\PY@do#1{\PY@bc{\PY@tc{\PY@ul{%
    \PY@it{\PY@bf{\PY@ff{#1}}}}}}}
\def\PY#1#2{\PY@reset\PY@toks#1+\relax+\PY@do{#2}}

\expandafter\def\csname PY@tok@w\endcsname{\def\PY@tc##1{\textcolor[rgb]{0.73,0.73,0.73}{##1}}}
\expandafter\def\csname PY@tok@c\endcsname{\let\PY@it=\textit\def\PY@tc##1{\textcolor[rgb]{0.25,0.50,0.50}{##1}}}
\expandafter\def\csname PY@tok@cp\endcsname{\def\PY@tc##1{\textcolor[rgb]{0.74,0.48,0.00}{##1}}}
\expandafter\def\csname PY@tok@k\endcsname{\let\PY@bf=\textbf\def\PY@tc##1{\textcolor[rgb]{0.00,0.50,0.00}{##1}}}
\expandafter\def\csname PY@tok@kp\endcsname{\def\PY@tc##1{\textcolor[rgb]{0.00,0.50,0.00}{##1}}}
\expandafter\def\csname PY@tok@kt\endcsname{\def\PY@tc##1{\textcolor[rgb]{0.69,0.00,0.25}{##1}}}
\expandafter\def\csname PY@tok@o\endcsname{\def\PY@tc##1{\textcolor[rgb]{0.40,0.40,0.40}{##1}}}
\expandafter\def\csname PY@tok@ow\endcsname{\let\PY@bf=\textbf\def\PY@tc##1{\textcolor[rgb]{0.67,0.13,1.00}{##1}}}
\expandafter\def\csname PY@tok@nb\endcsname{\def\PY@tc##1{\textcolor[rgb]{0.00,0.50,0.00}{##1}}}
\expandafter\def\csname PY@tok@nf\endcsname{\def\PY@tc##1{\textcolor[rgb]{0.00,0.00,1.00}{##1}}}
\expandafter\def\csname PY@tok@nc\endcsname{\let\PY@bf=\textbf\def\PY@tc##1{\textcolor[rgb]{0.00,0.00,1.00}{##1}}}
\expandafter\def\csname PY@tok@nn\endcsname{\let\PY@bf=\textbf\def\PY@tc##1{\textcolor[rgb]{0.00,0.00,1.00}{##1}}}
\expandafter\def\csname PY@tok@ne\endcsname{\let\PY@bf=\textbf\def\PY@tc##1{\textcolor[rgb]{0.82,0.25,0.23}{##1}}}
\expandafter\def\csname PY@tok@nv\endcsname{\def\PY@tc##1{\textcolor[rgb]{0.10,0.09,0.49}{##1}}}
\expandafter\def\csname PY@tok@no\endcsname{\def\PY@tc##1{\textcolor[rgb]{0.53,0.00,0.00}{##1}}}
\expandafter\def\csname PY@tok@nl\endcsname{\def\PY@tc##1{\textcolor[rgb]{0.63,0.63,0.00}{##1}}}
\expandafter\def\csname PY@tok@ni\endcsname{\let\PY@bf=\textbf\def\PY@tc##1{\textcolor[rgb]{0.60,0.60,0.60}{##1}}}
\expandafter\def\csname PY@tok@na\endcsname{\def\PY@tc##1{\textcolor[rgb]{0.49,0.56,0.16}{##1}}}
\expandafter\def\csname PY@tok@nt\endcsname{\let\PY@bf=\textbf\def\PY@tc##1{\textcolor[rgb]{0.00,0.50,0.00}{##1}}}
\expandafter\def\csname PY@tok@nd\endcsname{\def\PY@tc##1{\textcolor[rgb]{0.67,0.13,1.00}{##1}}}
\expandafter\def\csname PY@tok@s\endcsname{\def\PY@tc##1{\textcolor[rgb]{0.73,0.13,0.13}{##1}}}
\expandafter\def\csname PY@tok@sd\endcsname{\let\PY@it=\textit\def\PY@tc##1{\textcolor[rgb]{0.73,0.13,0.13}{##1}}}
\expandafter\def\csname PY@tok@si\endcsname{\let\PY@bf=\textbf\def\PY@tc##1{\textcolor[rgb]{0.73,0.40,0.53}{##1}}}
\expandafter\def\csname PY@tok@se\endcsname{\let\PY@bf=\textbf\def\PY@tc##1{\textcolor[rgb]{0.73,0.40,0.13}{##1}}}
\expandafter\def\csname PY@tok@sr\endcsname{\def\PY@tc##1{\textcolor[rgb]{0.73,0.40,0.53}{##1}}}
\expandafter\def\csname PY@tok@ss\endcsname{\def\PY@tc##1{\textcolor[rgb]{0.10,0.09,0.49}{##1}}}
\expandafter\def\csname PY@tok@sx\endcsname{\def\PY@tc##1{\textcolor[rgb]{0.00,0.50,0.00}{##1}}}
\expandafter\def\csname PY@tok@m\endcsname{\def\PY@tc##1{\textcolor[rgb]{0.40,0.40,0.40}{##1}}}
\expandafter\def\csname PY@tok@gh\endcsname{\let\PY@bf=\textbf\def\PY@tc##1{\textcolor[rgb]{0.00,0.00,0.50}{##1}}}
\expandafter\def\csname PY@tok@gu\endcsname{\let\PY@bf=\textbf\def\PY@tc##1{\textcolor[rgb]{0.50,0.00,0.50}{##1}}}
\expandafter\def\csname PY@tok@gd\endcsname{\def\PY@tc##1{\textcolor[rgb]{0.63,0.00,0.00}{##1}}}
\expandafter\def\csname PY@tok@gi\endcsname{\def\PY@tc##1{\textcolor[rgb]{0.00,0.63,0.00}{##1}}}
\expandafter\def\csname PY@tok@gr\endcsname{\def\PY@tc##1{\textcolor[rgb]{1.00,0.00,0.00}{##1}}}
\expandafter\def\csname PY@tok@ge\endcsname{\let\PY@it=\textit}
\expandafter\def\csname PY@tok@gs\endcsname{\let\PY@bf=\textbf}
\expandafter\def\csname PY@tok@gp\endcsname{\let\PY@bf=\textbf\def\PY@tc##1{\textcolor[rgb]{0.00,0.00,0.50}{##1}}}
\expandafter\def\csname PY@tok@go\endcsname{\def\PY@tc##1{\textcolor[rgb]{0.53,0.53,0.53}{##1}}}
\expandafter\def\csname PY@tok@gt\endcsname{\def\PY@tc##1{\textcolor[rgb]{0.00,0.27,0.87}{##1}}}
\expandafter\def\csname PY@tok@err\endcsname{\def\PY@bc##1{\setlength{\fboxsep}{0pt}\fcolorbox[rgb]{1.00,0.00,0.00}{1,1,1}{\strut ##1}}}
\expandafter\def\csname PY@tok@kc\endcsname{\let\PY@bf=\textbf\def\PY@tc##1{\textcolor[rgb]{0.00,0.50,0.00}{##1}}}
\expandafter\def\csname PY@tok@kd\endcsname{\let\PY@bf=\textbf\def\PY@tc##1{\textcolor[rgb]{0.00,0.50,0.00}{##1}}}
\expandafter\def\csname PY@tok@kn\endcsname{\let\PY@bf=\textbf\def\PY@tc##1{\textcolor[rgb]{0.00,0.50,0.00}{##1}}}
\expandafter\def\csname PY@tok@kr\endcsname{\let\PY@bf=\textbf\def\PY@tc##1{\textcolor[rgb]{0.00,0.50,0.00}{##1}}}
\expandafter\def\csname PY@tok@bp\endcsname{\def\PY@tc##1{\textcolor[rgb]{0.00,0.50,0.00}{##1}}}
\expandafter\def\csname PY@tok@fm\endcsname{\def\PY@tc##1{\textcolor[rgb]{0.00,0.00,1.00}{##1}}}
\expandafter\def\csname PY@tok@vc\endcsname{\def\PY@tc##1{\textcolor[rgb]{0.10,0.09,0.49}{##1}}}
\expandafter\def\csname PY@tok@vg\endcsname{\def\PY@tc##1{\textcolor[rgb]{0.10,0.09,0.49}{##1}}}
\expandafter\def\csname PY@tok@vi\endcsname{\def\PY@tc##1{\textcolor[rgb]{0.10,0.09,0.49}{##1}}}
\expandafter\def\csname PY@tok@vm\endcsname{\def\PY@tc##1{\textcolor[rgb]{0.10,0.09,0.49}{##1}}}
\expandafter\def\csname PY@tok@sa\endcsname{\def\PY@tc##1{\textcolor[rgb]{0.73,0.13,0.13}{##1}}}
\expandafter\def\csname PY@tok@sb\endcsname{\def\PY@tc##1{\textcolor[rgb]{0.73,0.13,0.13}{##1}}}
\expandafter\def\csname PY@tok@sc\endcsname{\def\PY@tc##1{\textcolor[rgb]{0.73,0.13,0.13}{##1}}}
\expandafter\def\csname PY@tok@dl\endcsname{\def\PY@tc##1{\textcolor[rgb]{0.73,0.13,0.13}{##1}}}
\expandafter\def\csname PY@tok@s2\endcsname{\def\PY@tc##1{\textcolor[rgb]{0.73,0.13,0.13}{##1}}}
\expandafter\def\csname PY@tok@sh\endcsname{\def\PY@tc##1{\textcolor[rgb]{0.73,0.13,0.13}{##1}}}
\expandafter\def\csname PY@tok@s1\endcsname{\def\PY@tc##1{\textcolor[rgb]{0.73,0.13,0.13}{##1}}}
\expandafter\def\csname PY@tok@mb\endcsname{\def\PY@tc##1{\textcolor[rgb]{0.40,0.40,0.40}{##1}}}
\expandafter\def\csname PY@tok@mf\endcsname{\def\PY@tc##1{\textcolor[rgb]{0.40,0.40,0.40}{##1}}}
\expandafter\def\csname PY@tok@mh\endcsname{\def\PY@tc##1{\textcolor[rgb]{0.40,0.40,0.40}{##1}}}
\expandafter\def\csname PY@tok@mi\endcsname{\def\PY@tc##1{\textcolor[rgb]{0.40,0.40,0.40}{##1}}}
\expandafter\def\csname PY@tok@il\endcsname{\def\PY@tc##1{\textcolor[rgb]{0.40,0.40,0.40}{##1}}}
\expandafter\def\csname PY@tok@mo\endcsname{\def\PY@tc##1{\textcolor[rgb]{0.40,0.40,0.40}{##1}}}
\expandafter\def\csname PY@tok@ch\endcsname{\let\PY@it=\textit\def\PY@tc##1{\textcolor[rgb]{0.25,0.50,0.50}{##1}}}
\expandafter\def\csname PY@tok@cm\endcsname{\let\PY@it=\textit\def\PY@tc##1{\textcolor[rgb]{0.25,0.50,0.50}{##1}}}
\expandafter\def\csname PY@tok@cpf\endcsname{\let\PY@it=\textit\def\PY@tc##1{\textcolor[rgb]{0.25,0.50,0.50}{##1}}}
\expandafter\def\csname PY@tok@c1\endcsname{\let\PY@it=\textit\def\PY@tc##1{\textcolor[rgb]{0.25,0.50,0.50}{##1}}}
\expandafter\def\csname PY@tok@cs\endcsname{\let\PY@it=\textit\def\PY@tc##1{\textcolor[rgb]{0.25,0.50,0.50}{##1}}}

\def\PYZbs{\char`\\}
\def\PYZus{\char`\_}
\def\PYZob{\char`\{}
\def\PYZcb{\char`\}}
\def\PYZca{\char`\^}
\def\PYZam{\char`\&}
\def\PYZlt{\char`\<}
\def\PYZgt{\char`\>}
\def\PYZsh{\char`\#}
\def\PYZpc{\char`\%}
\def\PYZdl{\char`\$}
\def\PYZhy{\char`\-}
\def\PYZsq{\char`\'}
\def\PYZdq{\char`\"}
\def\PYZti{\char`\~}
% for compatibility with earlier versions
\def\PYZat{@}
\def\PYZlb{[}
\def\PYZrb{]}
\makeatother


    % Exact colors from NB
    \definecolor{incolor}{rgb}{0.0, 0.0, 0.5}
    \definecolor{outcolor}{rgb}{0.545, 0.0, 0.0}



    
    % Prevent overflowing lines due to hard-to-break entities
    \sloppy 
    % Setup hyperref package
    \hypersetup{
      breaklinks=true,  % so long urls are correctly broken across lines
      colorlinks=true,
      urlcolor=urlcolor,
      linkcolor=linkcolor,
      citecolor=citecolor,
      }
    % Slightly bigger margins than the latex defaults
    
    \geometry{verbose,tmargin=1in,bmargin=1in,lmargin=1in,rmargin=1in}
    
    

    \begin{document}
    
    
    \maketitle
    
    

    
    \hypertarget{titanic---projeto-final}{%
\section{Titanic - Projeto Final}\label{titanic---projeto-final}}

\hypertarget{introduuxe7uxe3o}{%
\subsection{Introdução}\label{introduuxe7uxe3o}}

O acidente do Titanic é o mais famoso acidente marítimo já ocorrido.
Toda essa fama veio do filme Titanic de 1997. O acidente ocorreu no dia
15 de abril de 1915 e 1502 tripulantes dos 2224 que estavam a bordo
morreram. No entanto, alguns grupos de pessoas tiveram uma maior taxa de
sobrevivência. Sendo assim, neste projeto será analisado informações de
891 passageiros do RMS Titanic para extrairmos informações relevantes
sobre o ocorrido.

\hypertarget{variuxe1veis-do-conjunto-de-dados}{%
\subsubsection{Variáveis do conjunto de
dados}\label{variuxe1veis-do-conjunto-de-dados}}

\begin{longtable}[]{@{}lll@{}}
\toprule
Variável & Definição & Chave\tabularnewline
\midrule
\endhead
survival & Sobrevivencia & 0 = Não, 1=Sim\tabularnewline
pclass & Classe da passagem &
1=Primeira,2=Segunda,3=Terceira\tabularnewline
sex & Sexo &\tabularnewline
Age & Idade em anos &\tabularnewline
sibsp & \# de irmãos/esposx à bordo &\tabularnewline
parch & \# de pais/filho à bordo &\tabularnewline
fare & Preço da passagem &\tabularnewline
cabin & Número da cabine &\tabularnewline
embarked & Porto de embarque & C = Cherbourg,Q = Queenstown,
S=Southampton\tabularnewline
\bottomrule
\end{longtable}

    \begin{Verbatim}[commandchars=\\\{\}]
{\color{incolor}In [{\color{incolor}1}]:} \PY{c+c1}{\PYZsh{} Escreva aqui os imports, no mínimo você vai usar pandas, numpy e matplotlib }
        \PY{c+c1}{\PYZsh{} também pode\PYZhy{}se usar outras bibliotecas se desejar.}
        \PY{k+kn}{import} \PY{n+nn}{pandas} \PY{k}{as} \PY{n+nn}{pd}
        \PY{k+kn}{import} \PY{n+nn}{numpy} \PY{k}{as} \PY{n+nn}{np}
        \PY{k+kn}{import} \PY{n+nn}{matplotlib}\PY{n+nn}{.}\PY{n+nn}{pyplot} \PY{k}{as} \PY{n+nn}{plt}
        \PY{k+kn}{from} \PY{n+nn}{scipy} \PY{k}{import} \PY{n}{stats}
        
        \PY{c+c1}{\PYZsh{} Não esqueça que para o matplotlib plotar a baixo da célula é precisa adicionar}
        \PY{c+c1}{\PYZsh{} o código especial.}
        \PY{o}{\PYZpc{}}\PY{k}{matplotlib} inline
\end{Verbatim}


    \begin{Verbatim}[commandchars=\\\{\}]
{\color{incolor}In [{\color{incolor}2}]:} \PY{c+c1}{\PYZsh{} Abra o arquivo CSV e mostre algumas linhas}
        \PY{n}{df\PYZus{}titanic} \PY{o}{=} \PY{n}{pd}\PY{o}{.}\PY{n}{read\PYZus{}csv}\PY{p}{(}\PY{l+s+s1}{\PYZsq{}}\PY{l+s+s1}{titanic\PYZhy{}data\PYZhy{}6.csv}\PY{l+s+s1}{\PYZsq{}}\PY{p}{)}
        \PY{n}{df\PYZus{}titanic}\PY{o}{.}\PY{n}{head}\PY{p}{(}\PY{l+m+mi}{2}\PY{p}{)}
\end{Verbatim}


\begin{Verbatim}[commandchars=\\\{\}]
{\color{outcolor}Out[{\color{outcolor}2}]:}    PassengerId  Survived  Pclass  \textbackslash{}
        0            1         0       3   
        1            2         1       1   
        
                                                        Name     Sex   Age  SibSp  \textbackslash{}
        0                            Braund, Mr. Owen Harris    male  22.0      1   
        1  Cumings, Mrs. John Bradley (Florence Briggs Th{\ldots}  female  38.0      1   
        
           Parch     Ticket     Fare Cabin Embarked  
        0      0  A/5 21171   7.2500   NaN        S  
        1      0   PC 17599  71.2833   C85        C  
\end{Verbatim}
            
    \hypertarget{perguntas}{%
\subsection{Perguntas}\label{perguntas}}

\begin{itemize}
\tightlist
\item
  Mulheres e crianças tiveram maior taxa de sobrevivencia comparado com
  outros grupos?
\end{itemize}

Está pergunta foi baseada no fato que no filme Titanic o capitão ordena
que a evacuação de mulheres e crianças seja feita primeiro.

\begin{itemize}
\tightlist
\item
  Qual foi a relação entre a taxa de sobrevivencia com o número de
  parentes à bordo?
\end{itemize}

Parentes tendem a cuidar um do outro.

\begin{itemize}
\item
  Qual é a distribuição da tripulação por idade e a distribuição de
  sobreviventes por idade?
\item
  Quantidade preço médio para passagens para cada classe?
\item
  Qual a taxa de sobrevivencia dentro de cada classe?
\end{itemize}

Analisando se o aspecto financeiro influenciou na taxa de sobrevivencia.

    \hypertarget{limpeza-dos-dados}{%
\subsection{Limpeza dos dados}\label{limpeza-dos-dados}}

Primeiro passo é verificar o estado inicial dos dados

    \begin{Verbatim}[commandchars=\\\{\}]
{\color{incolor}In [{\color{incolor}3}]:} \PY{c+c1}{\PYZsh{} Verificação inicial}
        \PY{n}{df\PYZus{}titanic}\PY{o}{.}\PY{n}{info}\PY{p}{(}\PY{p}{)}
        \PY{c+c1}{\PYZsh{}Dados duplicados}
        \PY{n+nb}{print}\PY{p}{(}\PY{l+s+s1}{\PYZsq{}}\PY{l+s+si}{\PYZob{}\PYZcb{}}\PY{l+s+s1}{ linhas duplicadas}\PY{l+s+s1}{\PYZsq{}}\PY{o}{.}\PY{n}{format}\PY{p}{(}\PY{n}{df\PYZus{}titanic}\PY{o}{.}\PY{n}{duplicated}\PY{p}{(}\PY{p}{)}\PY{o}{.}\PY{n}{sum}\PY{p}{(}\PY{p}{)}\PY{p}{)}\PY{p}{)}
\end{Verbatim}


    \begin{Verbatim}[commandchars=\\\{\}]
<class 'pandas.core.frame.DataFrame'>
RangeIndex: 891 entries, 0 to 890
Data columns (total 12 columns):
PassengerId    891 non-null int64
Survived       891 non-null int64
Pclass         891 non-null int64
Name           891 non-null object
Sex            891 non-null object
Age            714 non-null float64
SibSp          891 non-null int64
Parch          891 non-null int64
Ticket         891 non-null object
Fare           891 non-null float64
Cabin          204 non-null object
Embarked       889 non-null object
dtypes: float64(2), int64(5), object(5)
memory usage: 83.6+ KB
0 linhas duplicadas

    \end{Verbatim}

    Pela verificação inicial podemos ver que: - Precisamos transformar os
nomes das colunas para minusculo - Existem dados faltantes nas colunas
\emph{Age},\emph{Cabin} e \emph{Embarked} - Eliminar colunas que não
serão usadas para o estudo

Primeiramente, transformaremos o nome das colunas para minusculo

    \begin{Verbatim}[commandchars=\\\{\}]
{\color{incolor}In [{\color{incolor}4}]:} \PY{n}{df\PYZus{}titanic} \PY{o}{=} \PY{n}{df\PYZus{}titanic}\PY{o}{.}\PY{n}{rename}\PY{p}{(}\PY{n+nb}{str}\PY{o}{.}\PY{n}{lower}\PY{p}{,} \PY{n}{axis}\PY{o}{=}\PY{l+s+s1}{\PYZsq{}}\PY{l+s+s1}{columns}\PY{l+s+s1}{\PYZsq{}}\PY{p}{)}
\end{Verbatim}


    Excluindo as colunas \emph{Ticket} e \emph{Cabin}, pois, no momento, não
coseguimos extrair informações de \emph{Ticket} e temos muitos dados
faltantes de \emph{Cabin}

    \begin{Verbatim}[commandchars=\\\{\}]
{\color{incolor}In [{\color{incolor}5}]:} \PY{n}{df\PYZus{}titanic} \PY{o}{=} \PY{n}{df\PYZus{}titanic}\PY{o}{.}\PY{n}{drop}\PY{p}{(}\PY{p}{[}\PY{l+s+s1}{\PYZsq{}}\PY{l+s+s1}{cabin}\PY{l+s+s1}{\PYZsq{}}\PY{p}{,}\PY{l+s+s1}{\PYZsq{}}\PY{l+s+s1}{ticket}\PY{l+s+s1}{\PYZsq{}}\PY{p}{]}\PY{p}{,}\PY{n}{axis}\PY{o}{=}\PY{l+s+s1}{\PYZsq{}}\PY{l+s+s1}{columns}\PY{l+s+s1}{\PYZsq{}}\PY{p}{)}
\end{Verbatim}


    Lidando com os valores nulos.

Para a coluna \emph{age} preencheremos os dados nulos com a media da
idade

Para a coluna \emph{embarked} preencheremos com a Moda

    \begin{Verbatim}[commandchars=\\\{\}]
{\color{incolor}In [{\color{incolor}6}]:} \PY{c+c1}{\PYZsh{} Preenchendo a coluna Age com a média}
        \PY{k}{def} \PY{n+nf}{fill\PYZus{}na\PYZus{}col\PYZus{}mean}\PY{p}{(}\PY{n}{df}\PY{p}{,}\PY{n}{column}\PY{p}{)}\PY{p}{:}
            \PY{l+s+sd}{\PYZdq{}\PYZdq{}\PYZdq{}}
        \PY{l+s+sd}{    Fill the NaN values of a column in a dataframe with the mean of the values}
        \PY{l+s+sd}{    of the own column}
        \PY{l+s+sd}{    Args:}
        \PY{l+s+sd}{        df(Pandas.DataFrame)}
        \PY{l+s+sd}{        column(String):Label of the column}
        \PY{l+s+sd}{    Returns:}
        \PY{l+s+sd}{        Pandas.DataFrame}
        \PY{l+s+sd}{    \PYZdq{}\PYZdq{}\PYZdq{}}
            \PY{n}{mean} \PY{o}{=} \PY{n}{np}\PY{o}{.}\PY{n}{rint}\PY{p}{(}\PY{n}{np}\PY{o}{.}\PY{n}{mean}\PY{p}{(}\PY{n}{df}\PY{p}{[}\PY{n}{column}\PY{p}{]}\PY{p}{)}\PY{p}{)}
            \PY{k}{return} \PY{n}{df}\PY{o}{.}\PY{n}{fillna}\PY{p}{(}\PY{n}{value}\PY{o}{=}\PY{p}{\PYZob{}}\PY{n}{column}\PY{p}{:}\PY{n}{mean}\PY{p}{\PYZcb{}}\PY{p}{)}
        
        \PY{n}{df\PYZus{}titanic} \PY{o}{=} \PY{n}{fill\PYZus{}na\PYZus{}col\PYZus{}mean}\PY{p}{(}\PY{n}{df\PYZus{}titanic}\PY{p}{,}\PY{l+s+s1}{\PYZsq{}}\PY{l+s+s1}{age}\PY{l+s+s1}{\PYZsq{}}\PY{p}{)}
\end{Verbatim}


    \begin{Verbatim}[commandchars=\\\{\}]
{\color{incolor}In [{\color{incolor}7}]:} \PY{c+c1}{\PYZsh{} Preenchendo a coluna embarked com a moda}
        \PY{k}{def} \PY{n+nf}{fill\PYZus{}na\PYZus{}col\PYZus{}mode}\PY{p}{(}\PY{n}{df}\PY{p}{,}\PY{n}{column}\PY{p}{)}\PY{p}{:}
            \PY{l+s+sd}{\PYZdq{}\PYZdq{}\PYZdq{}}
        \PY{l+s+sd}{    Fill the NaN values of a column in a dataframe with the mode of the values}
        \PY{l+s+sd}{    of the own column}
        \PY{l+s+sd}{    Args:}
        \PY{l+s+sd}{        df(Pandas.DataFrame)}
        \PY{l+s+sd}{        column(String):Label of the column}
        \PY{l+s+sd}{    Returns:}
        \PY{l+s+sd}{        Pandas.DataFrame}
        \PY{l+s+sd}{    \PYZdq{}\PYZdq{}\PYZdq{}}
            \PY{n}{mode} \PY{o}{=} \PY{n}{df}\PY{p}{[}\PY{n}{column}\PY{p}{]}\PY{o}{.}\PY{n}{mode}\PY{p}{(}\PY{p}{)}\PY{p}{[}\PY{l+m+mi}{0}\PY{p}{]}
            \PY{k}{return} \PY{n}{df}\PY{o}{.}\PY{n}{fillna}\PY{p}{(}\PY{n}{value}\PY{o}{=}\PY{p}{\PYZob{}}\PY{n}{column}\PY{p}{:}\PY{n}{mode}\PY{p}{\PYZcb{}}\PY{p}{)}
        
        \PY{n}{df\PYZus{}titanic} \PY{o}{=} \PY{n}{fill\PYZus{}na\PYZus{}col\PYZus{}mode}\PY{p}{(}\PY{n}{df\PYZus{}titanic}\PY{p}{,}\PY{l+s+s1}{\PYZsq{}}\PY{l+s+s1}{embarked}\PY{l+s+s1}{\PYZsq{}}\PY{p}{)}
\end{Verbatim}


    Verificando se os dados contem duplicatas

    \begin{Verbatim}[commandchars=\\\{\}]
{\color{incolor}In [{\color{incolor}8}]:} \PY{n}{df\PYZus{}titanic}\PY{o}{.}\PY{n}{duplicated}\PY{p}{(}\PY{p}{)}\PY{o}{.}\PY{n}{any}\PY{p}{(}\PY{p}{)}
\end{Verbatim}


\begin{Verbatim}[commandchars=\\\{\}]
{\color{outcolor}Out[{\color{outcolor}8}]:} False
\end{Verbatim}
            
    Verificando o estado final dos dados

    \begin{Verbatim}[commandchars=\\\{\}]
{\color{incolor}In [{\color{incolor}9}]:} \PY{n}{df\PYZus{}titanic}\PY{o}{.}\PY{n}{info}\PY{p}{(}\PY{p}{)}
\end{Verbatim}


    \begin{Verbatim}[commandchars=\\\{\}]
<class 'pandas.core.frame.DataFrame'>
RangeIndex: 891 entries, 0 to 890
Data columns (total 10 columns):
passengerid    891 non-null int64
survived       891 non-null int64
pclass         891 non-null int64
name           891 non-null object
sex            891 non-null object
age            891 non-null float64
sibsp          891 non-null int64
parch          891 non-null int64
fare           891 non-null float64
embarked       891 non-null object
dtypes: float64(2), int64(5), object(3)
memory usage: 69.7+ KB

    \end{Verbatim}

    Finalizado a limpeza podemos salvar os dados

    \begin{Verbatim}[commandchars=\\\{\}]
{\color{incolor}In [{\color{incolor}10}]:} \PY{n}{df\PYZus{}titanic}\PY{o}{.}\PY{n}{to\PYZus{}csv}\PY{p}{(}\PY{l+s+s1}{\PYZsq{}}\PY{l+s+s1}{titanic\PYZus{}clean.csv}\PY{l+s+s1}{\PYZsq{}}\PY{p}{,}\PY{n}{index}\PY{o}{=}\PY{k+kc}{False}\PY{p}{)}
\end{Verbatim}


    \hypertarget{anuxe1lise}{%
\subsection{Análise}\label{anuxe1lise}}

Fale um pouco do processo e forma que pretende analisar os dados, também
use pelo menos dois tipos diferentes de gráficos.

O site do matplotlib é cheio de exemplos que você pode seguir e usar em
seu projeto. Também não é preciso se limitar apenas ao matplotlib,
existem outras bibliotecas gráficas, um exemplo é o seaborn.

Ao final de análisar cada dado e figura gerada faça uma explicação dos
resultados respondendo a sua pergunta. Tente falar de forma simples mas
completa.

    \begin{Verbatim}[commandchars=\\\{\}]
{\color{incolor}In [{\color{incolor}11}]:} \PY{c+c1}{\PYZsh{} Lendo o arquivo CSV limpo}
         
         \PY{n}{df\PYZus{}titanic} \PY{o}{=} \PY{n}{pd}\PY{o}{.}\PY{n}{read\PYZus{}csv}\PY{p}{(}\PY{l+s+s1}{\PYZsq{}}\PY{l+s+s1}{titanic\PYZus{}clean.csv}\PY{l+s+s1}{\PYZsq{}}\PY{p}{,}\PY{n}{index\PYZus{}col}\PY{o}{=}\PY{l+m+mi}{0}\PY{p}{)}
         \PY{n}{df\PYZus{}titanic}\PY{o}{.}\PY{n}{head}\PY{p}{(}\PY{l+m+mi}{2}\PY{p}{)}
\end{Verbatim}


\begin{Verbatim}[commandchars=\\\{\}]
{\color{outcolor}Out[{\color{outcolor}11}]:}              survived  pclass  \textbackslash{}
         passengerid                     
         1                   0       3   
         2                   1       1   
         
                                                                   name     sex   age  \textbackslash{}
         passengerid                                                                    
         1                                      Braund, Mr. Owen Harris    male  22.0   
         2            Cumings, Mrs. John Bradley (Florence Briggs Th{\ldots}  female  38.0   
         
                      sibsp  parch     fare embarked  
         passengerid                                  
         1                1      0   7.2500        S  
         2                1      0  71.2833        C  
\end{Verbatim}
            
    \hypertarget{p1.mulheres-e-crianuxe7as-tiveram-maior-taxa-de-sobrevivencia-comparado-com-outros-grupos}{%
\subsubsection{P1.Mulheres e crianças tiveram maior taxa de
sobrevivencia comparado com outros
grupos?}\label{p1.mulheres-e-crianuxe7as-tiveram-maior-taxa-de-sobrevivencia-comparado-com-outros-grupos}}

\textbf{Obs.:} De acordo com os Direitos da Criança da ONU, uma criança
é qualquer individuo com idade menor que 18 anos

    \begin{Verbatim}[commandchars=\\\{\}]
{\color{incolor}In [{\color{incolor}12}]:} \PY{c+c1}{\PYZsh{} Separando o conjunto de mulheres e criancas}
         \PY{n}{df\PYZus{}wm\PYZus{}ch} \PY{o}{=} \PY{n}{df\PYZus{}titanic}\PY{o}{.}\PY{n}{query}\PY{p}{(}\PY{l+s+s1}{\PYZsq{}}\PY{l+s+s1}{sex == }\PY{l+s+s1}{\PYZdq{}}\PY{l+s+s1}{female}\PY{l+s+s1}{\PYZdq{}}\PY{l+s+s1}{ or age \PYZlt{} 18}\PY{l+s+s1}{\PYZsq{}}\PY{p}{)}
\end{Verbatim}


    \begin{Verbatim}[commandchars=\\\{\}]
{\color{incolor}In [{\color{incolor}13}]:} \PY{c+c1}{\PYZsh{}funcao para calcular a taxa de sobrevivencia}
         \PY{k}{def} \PY{n+nf}{surv\PYZus{}rate}\PY{p}{(}\PY{n}{df}\PY{p}{)}\PY{p}{:}
             \PY{l+s+sd}{\PYZdq{}\PYZdq{}\PYZdq{}}
         \PY{l+s+sd}{    Find the survival rate of a dataframe of the Kaggle Titanic dataset}
         \PY{l+s+sd}{    Args:}
         \PY{l+s+sd}{        df(Pandas.DataFrame)}
         \PY{l+s+sd}{    Returns:}
         \PY{l+s+sd}{        float with the survivale rate}
         \PY{l+s+sd}{    \PYZdq{}\PYZdq{}\PYZdq{}}
             \PY{n}{surv\PYZus{}group} \PY{o}{=} \PY{n}{df}\PY{o}{.}\PY{n}{query}\PY{p}{(}\PY{l+s+s1}{\PYZsq{}}\PY{l+s+s1}{survived == 1}\PY{l+s+s1}{\PYZsq{}}\PY{p}{)}
             \PY{k}{return} \PY{p}{(}\PY{n}{surv\PYZus{}group}\PY{o}{.}\PY{n}{shape}\PY{p}{[}\PY{l+m+mi}{0}\PY{p}{]}\PY{o}{/}\PY{n}{df}\PY{o}{.}\PY{n}{shape}\PY{p}{[}\PY{l+m+mi}{0}\PY{p}{]}\PY{p}{)}\PY{o}{*}\PY{l+m+mi}{100}
\end{Verbatim}


    \begin{Verbatim}[commandchars=\\\{\}]
{\color{incolor}In [{\color{incolor}14}]:} \PY{c+c1}{\PYZsh{}Comparação}
         \PY{n+nb}{print}\PY{p}{(}\PY{l+s+s1}{\PYZsq{}}\PY{l+s+si}{\PYZob{}:.2f\PYZcb{}}\PY{l+s+s1}{ porcento das pessoas sobreviveram ao acidente}\PY{l+s+s1}{\PYZsq{}}\PY{o}{.}\PY{n}{format}\PY{p}{(}\PY{n}{surv\PYZus{}rate}\PY{p}{(}\PY{n}{df\PYZus{}titanic}\PY{p}{)}\PY{p}{)}\PY{p}{)}
         \PY{n+nb}{print}\PY{p}{(}\PY{l+s+s1}{\PYZsq{}}\PY{l+s+si}{\PYZob{}:.2f\PYZcb{}}\PY{l+s+s1}{ porcento das mulheres e crianças sobreviveram ao acidente}\PY{l+s+s1}{\PYZsq{}}\PY{o}{.}\PY{n}{format}\PY{p}{(}\PY{n}{surv\PYZus{}rate}\PY{p}{(}\PY{n}{df\PYZus{}wm\PYZus{}ch}\PY{p}{)}\PY{p}{)}\PY{p}{)}
\end{Verbatim}


    \begin{Verbatim}[commandchars=\\\{\}]
38.38 porcento das pessoas sobreviveram ao acidente
68.82 porcento das mulheres e crianças sobreviveram ao acidente

    \end{Verbatim}

    \hypertarget{p2.qual-foi-a-relauxe7uxe3o-entre-a-taxa-de-sobrevivencia-com-o-nuxfamero-de-parentes-uxe0-bordo}{%
\subsubsection{P2.Qual foi a relação entre a taxa de sobrevivencia com o
número de parentes à
bordo?}\label{p2.qual-foi-a-relauxe7uxe3o-entre-a-taxa-de-sobrevivencia-com-o-nuxfamero-de-parentes-uxe0-bordo}}

O número de parentes será considerado como a soma das colunas
\emph{sibsp} e \emph{parch}

    \begin{Verbatim}[commandchars=\\\{\}]
{\color{incolor}In [{\color{incolor}15}]:} \PY{c+c1}{\PYZsh{}Criando a coluna que representa a soma}
         \PY{n}{df\PYZus{}titanic}\PY{p}{[}\PY{l+s+s1}{\PYZsq{}}\PY{l+s+s1}{relatives}\PY{l+s+s1}{\PYZsq{}}\PY{p}{]} \PY{o}{=} \PY{n}{df\PYZus{}titanic}\PY{p}{[}\PY{l+s+s1}{\PYZsq{}}\PY{l+s+s1}{sibsp}\PY{l+s+s1}{\PYZsq{}}\PY{p}{]} \PY{o}{+} \PY{n}{df\PYZus{}titanic}\PY{p}{[}\PY{l+s+s1}{\PYZsq{}}\PY{l+s+s1}{parch}\PY{l+s+s1}{\PYZsq{}}\PY{p}{]}
\end{Verbatim}


    \begin{Verbatim}[commandchars=\\\{\}]
{\color{incolor}In [{\color{incolor}16}]:} \PY{c+c1}{\PYZsh{}Grafico de barras da taxa de sobreviventes por quantidade de parentes}
         \PY{c+c1}{\PYZsh{}Calculando as quantidades relativas}
         \PY{n}{survivors} \PY{o}{=} \PY{n}{df\PYZus{}titanic}\PY{o}{.}\PY{n}{query}\PY{p}{(}\PY{l+s+s1}{\PYZsq{}}\PY{l+s+s1}{survived == 1}\PY{l+s+s1}{\PYZsq{}}\PY{p}{)}
         \PY{n}{surv\PYZus{}per\PYZus{}relative} \PY{o}{=} \PY{n}{survivors}\PY{p}{[}\PY{l+s+s1}{\PYZsq{}}\PY{l+s+s1}{relatives}\PY{l+s+s1}{\PYZsq{}}\PY{p}{]}\PY{o}{.}\PY{n}{value\PYZus{}counts}\PY{p}{(}\PY{p}{)}
         \PY{n}{total\PYZus{}per\PYZus{}relative} \PY{o}{=} \PY{n}{df\PYZus{}titanic}\PY{p}{[}\PY{l+s+s1}{\PYZsq{}}\PY{l+s+s1}{relatives}\PY{l+s+s1}{\PYZsq{}}\PY{p}{]}\PY{o}{.}\PY{n}{value\PYZus{}counts}\PY{p}{(}\PY{p}{)}
         \PY{n+nb}{print}\PY{p}{(}\PY{n}{total\PYZus{}per\PYZus{}relative}\PY{p}{)}
         \PY{n}{rate\PYZus{}relatives} \PY{o}{=} \PY{n}{surv\PYZus{}per\PYZus{}relative}\PY{o}{/}\PY{n}{total\PYZus{}per\PYZus{}relative}
         \PY{n+nb}{print}\PY{p}{(}\PY{n}{rate\PYZus{}relatives}\PY{p}{)}
         \PY{c+c1}{\PYZsh{}Precisamos substituir os valores NaN por zero}
         \PY{n}{rate\PYZus{}relatives} \PY{o}{=} \PY{n}{rate\PYZus{}relatives}\PY{o}{.}\PY{n}{fillna}\PY{p}{(}\PY{l+m+mi}{0}\PY{p}{)}
\end{Verbatim}


    \begin{Verbatim}[commandchars=\\\{\}]
0     537
1     161
2     102
3      29
5      22
4      15
6      12
10      7
7       6
Name: relatives, dtype: int64
0     0.303538
1     0.552795
2     0.578431
3     0.724138
4     0.200000
5     0.136364
6     0.333333
7          NaN
10         NaN
Name: relatives, dtype: float64

    \end{Verbatim}

    \begin{Verbatim}[commandchars=\\\{\}]
{\color{incolor}In [{\color{incolor}17}]:} \PY{c+c1}{\PYZsh{}Plotando grafico}
         \PY{n}{plt}\PY{o}{.}\PY{n}{bar}\PY{p}{(}\PY{n}{rate\PYZus{}relatives}\PY{o}{.}\PY{n}{index}\PY{p}{,}\PY{n}{rate\PYZus{}relatives}\PY{o}{.}\PY{n}{values}\PY{p}{)}
         \PY{n}{plt}\PY{o}{.}\PY{n}{xlabel}\PY{p}{(}\PY{l+s+s1}{\PYZsq{}}\PY{l+s+s1}{Quantidade de parentes}\PY{l+s+s1}{\PYZsq{}}\PY{p}{)}
         \PY{n}{plt}\PY{o}{.}\PY{n}{ylabel}\PY{p}{(}\PY{l+s+s1}{\PYZsq{}}\PY{l+s+s1}{Taxa de sobrevivencia}\PY{l+s+s1}{\PYZsq{}}\PY{p}{)}
         \PY{n}{plt}\PY{o}{.}\PY{n}{xticks}\PY{p}{(}\PY{n}{rate\PYZus{}relatives}\PY{o}{.}\PY{n}{index}\PY{p}{)}
         \PY{n}{plt}\PY{o}{.}\PY{n}{title}\PY{p}{(}\PY{l+s+s1}{\PYZsq{}}\PY{l+s+s1}{Taxa de sobrevivencia por quantidade de parentes}\PY{l+s+s1}{\PYZsq{}}\PY{p}{)}\PY{p}{;}
\end{Verbatim}


    \begin{center}
    \adjustimage{max size={0.9\linewidth}{0.9\paperheight}}{output_28_0.png}
    \end{center}
    { \hspace*{\fill} \\}
    
    \hypertarget{p3.qual-uxe9-a-distribuiuxe7uxe3o-da-tripulauxe7uxe3o-por-idade-e-a-distribuiuxe7uxe3o-de-sobreviventes-por-idade}{%
\subsubsection{P3.Qual é a distribuição da tripulação por idade e a
distribuição de sobreviventes por
idade?}\label{p3.qual-uxe9-a-distribuiuxe7uxe3o-da-tripulauxe7uxe3o-por-idade-e-a-distribuiuxe7uxe3o-de-sobreviventes-por-idade}}

    \begin{Verbatim}[commandchars=\\\{\}]
{\color{incolor}In [{\color{incolor}18}]:} \PY{c+c1}{\PYZsh{}Plotando a distribuição por idade}
         \PY{n}{df\PYZus{}titanic}\PY{p}{[}\PY{l+s+s1}{\PYZsq{}}\PY{l+s+s1}{age}\PY{l+s+s1}{\PYZsq{}}\PY{p}{]}\PY{o}{.}\PY{n}{hist}\PY{p}{(}\PY{n}{label}\PY{o}{=}\PY{l+s+s1}{\PYZsq{}}\PY{l+s+s1}{Tripulação inicial}\PY{l+s+s1}{\PYZsq{}}\PY{p}{)}
         \PY{c+c1}{\PYZsh{}Plotando a distribuição de sobreviventes por idade}
         \PY{n}{df\PYZus{}titanic}\PY{o}{.}\PY{n}{query}\PY{p}{(}\PY{l+s+s1}{\PYZsq{}}\PY{l+s+s1}{survived == 1}\PY{l+s+s1}{\PYZsq{}}\PY{p}{)}\PY{p}{[}\PY{l+s+s1}{\PYZsq{}}\PY{l+s+s1}{age}\PY{l+s+s1}{\PYZsq{}}\PY{p}{]}\PY{o}{.}\PY{n}{hist}\PY{p}{(}\PY{n}{label}\PY{o}{=}\PY{l+s+s1}{\PYZsq{}}\PY{l+s+s1}{Sobreviventes}\PY{l+s+s1}{\PYZsq{}}\PY{p}{)}
         \PY{n}{plt}\PY{o}{.}\PY{n}{xlabel}\PY{p}{(}\PY{l+s+s1}{\PYZsq{}}\PY{l+s+s1}{Idade}\PY{l+s+s1}{\PYZsq{}}\PY{p}{)}
         \PY{n}{plt}\PY{o}{.}\PY{n}{ylabel}\PY{p}{(}\PY{l+s+s1}{\PYZsq{}}\PY{l+s+s1}{Quantidade}\PY{l+s+s1}{\PYZsq{}}\PY{p}{)}
         \PY{n}{plt}\PY{o}{.}\PY{n}{legend}\PY{p}{(}\PY{p}{)}
         \PY{n}{plt}\PY{o}{.}\PY{n}{title}\PY{p}{(}\PY{l+s+s1}{\PYZsq{}}\PY{l+s+s1}{Distribuição das pessoas antes e depois do acidente}\PY{l+s+s1}{\PYZsq{}}\PY{p}{)}\PY{p}{;}
\end{Verbatim}


    \begin{center}
    \adjustimage{max size={0.9\linewidth}{0.9\paperheight}}{output_30_0.png}
    \end{center}
    { \hspace*{\fill} \\}
    
    \hypertarget{p4.quantidade-preuxe7o-muxe9dio-para-passagens-para-cada-classe}{%
\subsubsection{P4.Quantidade preço médio para passagens para cada
classe?}\label{p4.quantidade-preuxe7o-muxe9dio-para-passagens-para-cada-classe}}

    \begin{Verbatim}[commandchars=\\\{\}]
{\color{incolor}In [{\color{incolor}19}]:} \PY{c+c1}{\PYZsh{}Encontrando o preço por class}
         \PY{n}{price\PYZus{}class} \PY{o}{=} \PY{n}{df\PYZus{}titanic}\PY{o}{.}\PY{n}{groupby}\PY{p}{(}\PY{l+s+s1}{\PYZsq{}}\PY{l+s+s1}{pclass}\PY{l+s+s1}{\PYZsq{}}\PY{p}{)}\PY{o}{.}\PY{n}{mean}\PY{p}{(}\PY{p}{)}\PY{p}{[}\PY{l+s+s1}{\PYZsq{}}\PY{l+s+s1}{fare}\PY{l+s+s1}{\PYZsq{}}\PY{p}{]}
         \PY{k}{for} \PY{n}{classe}\PY{p}{,}\PY{n}{preco} \PY{o+ow}{in} \PY{n}{price\PYZus{}class}\PY{o}{.}\PY{n}{iteritems}\PY{p}{(}\PY{p}{)}\PY{p}{:}
             \PY{n+nb}{print}\PY{p}{(}\PY{l+s+s1}{\PYZsq{}}\PY{l+s+s1}{O preço medio para a }\PY{l+s+si}{\PYZob{}\PYZcb{}}\PY{l+s+s1}{ classe é de }\PY{l+s+si}{\PYZob{}:.2f\PYZcb{}}\PY{l+s+s1}{ dolares.}\PY{l+s+s1}{\PYZsq{}}\PY{o}{.}\PY{n}{format}\PY{p}{(}\PY{n}{classe}\PY{p}{,}\PY{n}{preco}\PY{p}{)}\PY{p}{)}
\end{Verbatim}


    \begin{Verbatim}[commandchars=\\\{\}]
O preço medio para a 1 classe é de 84.15 dolares.
O preço medio para a 2 classe é de 20.66 dolares.
O preço medio para a 3 classe é de 13.68 dolares.

    \end{Verbatim}

    \hypertarget{p5.qual-a-taxa-de-sobrevivencia-dentro-de-cada-classe}{%
\subsubsection{P5.Qual a taxa de sobrevivencia dentro de cada
classe?}\label{p5.qual-a-taxa-de-sobrevivencia-dentro-de-cada-classe}}

    \begin{Verbatim}[commandchars=\\\{\}]
{\color{incolor}In [{\color{incolor}20}]:} \PY{c+c1}{\PYZsh{}Separando os dados por class}
         \PY{n}{df\PYZus{}primeira} \PY{o}{=} \PY{n}{df\PYZus{}titanic}\PY{o}{.}\PY{n}{query}\PY{p}{(}\PY{l+s+s1}{\PYZsq{}}\PY{l+s+s1}{pclass == 1}\PY{l+s+s1}{\PYZsq{}}\PY{p}{)}
         \PY{n}{df\PYZus{}segunda} \PY{o}{=} \PY{n}{df\PYZus{}titanic}\PY{o}{.}\PY{n}{query}\PY{p}{(}\PY{l+s+s1}{\PYZsq{}}\PY{l+s+s1}{pclass == 2}\PY{l+s+s1}{\PYZsq{}}\PY{p}{)}
         \PY{n}{df\PYZus{}terceira} \PY{o}{=} \PY{n}{df\PYZus{}titanic}\PY{o}{.}\PY{n}{query}\PY{p}{(}\PY{l+s+s1}{\PYZsq{}}\PY{l+s+s1}{pclass == 3}\PY{l+s+s1}{\PYZsq{}}\PY{p}{)}
         \PY{c+c1}{\PYZsh{}Mostrando os resultados}
         \PY{n+nb}{print}\PY{p}{(}\PY{l+s+s1}{\PYZsq{}}\PY{l+s+s1}{A taxa de sobrevivencia para a primeira class é de }\PY{l+s+si}{\PYZob{}:.2f\PYZcb{}}\PY{l+s+s1}{ porcento}\PY{l+s+s1}{\PYZsq{}}\PY{o}{.}\PY{n}{format}\PY{p}{(}\PY{n}{surv\PYZus{}rate}\PY{p}{(}\PY{n}{df\PYZus{}primeira}\PY{p}{)}\PY{p}{)}\PY{p}{)}
         \PY{n+nb}{print}\PY{p}{(}\PY{l+s+s1}{\PYZsq{}}\PY{l+s+s1}{A taxa de sobrevivencia para a segunda class é de }\PY{l+s+si}{\PYZob{}:.2f\PYZcb{}}\PY{l+s+s1}{ porcento}\PY{l+s+s1}{\PYZsq{}}\PY{o}{.}\PY{n}{format}\PY{p}{(}\PY{n}{surv\PYZus{}rate}\PY{p}{(}\PY{n}{df\PYZus{}segunda}\PY{p}{)}\PY{p}{)}\PY{p}{)}
         \PY{n+nb}{print}\PY{p}{(}\PY{l+s+s1}{\PYZsq{}}\PY{l+s+s1}{A taxa de sobrevivencia para a terceira class é de }\PY{l+s+si}{\PYZob{}:.2f\PYZcb{}}\PY{l+s+s1}{ porcento}\PY{l+s+s1}{\PYZsq{}}\PY{o}{.}\PY{n}{format}\PY{p}{(}\PY{n}{surv\PYZus{}rate}\PY{p}{(}\PY{n}{df\PYZus{}terceira}\PY{p}{)}\PY{p}{)}\PY{p}{)}
\end{Verbatim}


    \begin{Verbatim}[commandchars=\\\{\}]
A taxa de sobrevivencia para a primeira class é de 62.96 porcento
A taxa de sobrevivencia para a segunda class é de 47.28 porcento
A taxa de sobrevivencia para a terceira class é de 24.24 porcento

    \end{Verbatim}

    \hypertarget{resultados}{%
\subsection{Resultados}\label{resultados}}

P1:

Os dados obtidos comprovam o que foi mostrado no filme que mulheres e
crianças tiveram uma maior taxa de sobrevivencia.

P2:

A estudo mostra que pessoas que estavam viajando sozinha tiveram uma
taxa de sobrevivencia de, aproximadamente, 30\%. Enquanto pessoas que
estavam acompanhadas por parentes tiveram uma taxa de sobrevivencia de
até o dobro da taxa das pessoa que estavam sozinha. No entanto, quando o
numero de parentes passa de 3 a taxa volta a baixar. Isto deve ter
ocorrido devido a dificuldade de acomodar grandes grupos.

P3:

A distribuição das pessoas antes e depois do acidente mostra que a maior
parte das mortes ocorreram entre os adultos de, aproximadamente, 30
anos.

P4.

Cheguei a conclusão de que o valor estava em dólares atraves de uma
pesquisa na internet.

P5.

O estudo mostra que a condição financeira influenciou na taxa de
sobrevivencia dos passageiros.Uma vez que a taxa de sobrevivencia dos
passageiros de primeira classe foi quase 3 vezes maior que a de
passageiros de terceira classe.

\hypertarget{problemas-encontrados}{%
\subsubsection{Problemas encontrados}\label{problemas-encontrados}}

Um dos problemas encontrados foi como lidar com os valores ausentes da
coluna da idade, que alterou a precisão do estudo feito para responder a
pergunta 1.

Porém o maior dos problema é a falta de ferramentas para tirar
conclusões dos dados, pois neste estudo foi utilizado somente
estatística descritiva.

    \hypertarget{bibliografia}{%
\subsection{Bibliografia}\label{bibliografia}}

\href{https://guides.github.com/features/mastering-markdown/}{Mastering
Markdown}

\href{https://www.kaggle.com/c/titanic/data}{Kaggle Titanic}

\href{http://pandas.pydata.org/pandas-docs/stable/}{Pandas
Documentation}

\href{https://docs.scipy.org/doc/numpy-1.16.1/reference/}{Numpy
References}

\href{https://matplotlib.org/gallery/index.html}{Matplotlib Exemples}

\href{https://stackoverflow.com/}{StackOverflow}

\href{https://www.quora.com/What-were-the-ticket-prices-to-board-the-Titanic}{Quora}

\href{https://en.wikipedia.org/wiki/RMS_Titanic}{Wikipedia}


    % Add a bibliography block to the postdoc
    
    
    
    \end{document}
